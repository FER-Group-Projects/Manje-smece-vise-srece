\chapter{Zaključak i budući rad}
		
		% \textbf{\textit{dio 2. revizije}}\\
		
		 % \textit{U ovom poglavlju potrebno je napisati osvrt na vrijeme izrade projektnog zadatka, koji su tehnički izazovi prepoznati, jesu li riješeni ili kako bi mogli biti riješeni, koja su znanja stečena pri izradi projekta, koja bi znanja bila posebno potrebna za brže i kvalitetnije ostvarenje projekta i koje bi bile perspektive za nastavak rada u projektnoj grupi.}
		
		 % \textit{Potrebno je točno popisati funkcionalnosti koje nisu implementirane u ostvarenoj aplikaciji.}

		Zadatak naše grupe bio je razvoj web aplikacije za efikasniji i transparentniji odvoz smeća koja se postiže dodavanjem senzora na kontejnere te mogućnosti recenziranja istih kako bi se doznalo njihovo stanje i urednost. Izrada aplikacije (te popratne dokumentacije) trajala je 17 tjedana i bila je podijeljena u dvije faze.
		
		Prva faza započela je okupljanjem tima za razvoj aplikacije, ali većina vremena utrošena je na planiranje razvoja i dokumentaciju. Znanje stečeno na ovom predmetu pomoglo nam je pri izradi dokumentacije, što je uvelike pomoglo pri bržoj i efikasnijoj izradi same aplikacije. Ogroman broj raznih dijagrama koji variraju po kompleksnosti, veličini i funkciji pružao je pogled iz svih perspektiva, kako korisničkih, tako i sistemskih, na cijelu aplikaciju. Radi bolje koordinacije, timovi su bili podijeljeni u podskupine koje su, ovisno o fazi projekta, imale razne zadatke, kao što su razvoj frontenda i backenda. 
		
		Druga faza bila je nešto kraća od prve, ali činjenica da je trajala kraće ne znači da je bila lakša, dapače, tempo rada bio je puno intenzivniji jer je u pitanju bio sam razvoj aplikacije, što je (za većinu članova tima) uključivalo i učenje novih tehnologija i alata. Druga faza uključivala je i završetak projektne dokumentacije koja je, unatoč tome što najveći fokus nije bio na njoj, uspješno odrađena.
		
		Komunikacija među članovima vršila se pomoću aplikacije Slack koja je bila izuzetno korisna jer se moglo praktično i brzo podijeliti zadatke na sve članove tima.
		
		Iako se aplikacija zadržava na području jednog grada - Zagreba, moguće je buduće proširenje na druge gradove Hrvatske, ali i šire. 
		
		Sudjelovanje u izradi ove aplikacije bilo je neprocjenjivo iskustvo za sve članove razvojnog tima. Naučili smo raditi u raznim novim tehnologijama te smo shvatili koliki je vremenski (ali i timski) pothvat izrada ovakve aplikacije. Koordiniranost među članovima ekipe bila je izuzetno bitan faktor koji je omogućio da veći dio izrade aplikacije i dokumentacije prođe brzo i bez većih problema. Zadovoljni smo krajnjim rezultatom i, iako uvijek postoji prostor za poboljšanje, možemo reći da nam je svima drago što smo radili zajedno i što smo uspjeli postići ovakav monumentalni pothvat kao što je razvoj aplikacije.
		
		
		\eject 
