\chapter{Implementacija i korisničko sučelje}
		
		
		\section{Korištene tehnologije i alati}
		
%			\textbf{\textit{dio 2. revizije}}
%			
%			 \textit{Detaljno navesti sve tehnologije i alate koji su primijenjeni pri izradi dokumentacije i aplikacije. Ukratko ih opisati, te navesti njihovo značenje i mjesto primjene. Za svaki navedeni alat i tehnologiju je potrebno \textbf{navesti internet poveznicu} gdje se mogu preuzeti ili više saznati o njima}.
			
			Komunikacija u timu realizirana je korištenjem aplikacije \underline{Slack}\footnote{\url{https://slack.com/}}. Slack omogućuje laku podjelu u podskupine, brz način komuniciranja kao i mogućnost povezivanja s web platformom na kojoj se nalazi projekt kako bi se u svakom trenutku mogle vidjeti nove promjene na projektu. Za izradu UML dijagrama korišten je alat \underline{Astah Professional}\footnote{\url{http://astah.net/editions/professional}} koji služi za vizualizaciju ideja i stvaranje dijagrama koji pomažu pri lakšem razvoju i razumijevanju projektnog zadatka. Za sustav za upravljanje izvornim kodom korišten je \underline{Git}\footnote{\url{https://git-scm.com/}} koji je najpopularniji sustav za upravljanje izvornim kodom na svijetu. Udaljeni repozitorij projekta dostupan je na web platformi \underline{Gitlab}\footnote{\url{https://gitlab.com/}} koja služi za upravljanje git repozitorijima na centraliziranom serveru. Gitlab omogućuje potpunu kontrolu nad repozitorijima i članovi tima mogu odlučiti hoće li ti repozitoriji biti javno dostupni ili ne.
			
			Kao razvojno okruženje korišten je \underline{Eclipse}\footnote{\url{https://www.eclipse.org/}} - integrirano razvojno okruženje (IDE) tvrtke Eclipse Foundation. Služi prventstveno za razvoj aplikacija u Javi, ali aplikacije se mogu raditi i u drugim programskim jezicima pomoću plug-ina.
			
			Aplikacija je napisana koristeći radni okvir \underline{Spring Boot}\footnote{\url{https://spring.io/projects/spring-boot}} i jezik \underline{Java}\footnote{\url{https://www.java.com/en/}} za izradu backenda. Radni okvir Spring Boot ima mikro servis arhitekturu koja omogućuje da svaki servis ima svoj vlastiti proces što je vrlo korisno pri razvoju aplikacija. Java je brz, siguran i pouzdan objektno orijentirani programski jezik kojeg je dizajnirala tvrtka Sun Mircosystems. Za izradu frontenda korišteni su \underline{React}\footnote{\url{https://reactjs.org/}} i jezik \underline{JavaScript}\footnote{\url{https://www.javascript.com/}}. React je biblioteka u JavaScriptu za izgradnju korisničkih sučelja. Posjeduje ga tvrtka Facebook koja ga također i održava, a najčešće koristi kao osnova u razvoju web ili mobilnih aplikacija. Javascript je skriptno programski jezik koji se izvršava u web pregledniku na strani korisnika.
			
			Baza podataka nalazi se na poslužitelju u oblaku \underline{Heroku}\footnote{\url{https://www.heroku.com/}}. Heroku je cloud platforma koja omogućuje tvrtkama da izgrađuju, dostavljaju, nadziru i skaliraju aplikacije.
			
			\eject 

		
	
		\section{Ispitivanje programskog rješenja}
%			\textbf{\textit{dio 2. revizije}}\\
%			
%			 \textit{U ovom poglavlju je potrebno opisati provedbu ispitivanja implementiranih funkcionalnosti na razini komponenti i na razini cijelog sustava s prikazom odabranih ispitnih slučajeva. Studenti trebaju ispitati temeljnu funkcionalnost i rubne uvjete.}
			Sustav se ispituje jediničnim testovima \engl{unit tests} i integracijskim testovima \engl{integration tests}. 
			Jediničnim testovima se testiraju komponente backenda, dok se integracijskim testovima testira frontend i obrasci uporabe.
			
			U sljedeća dva potpoglavlja razloženi su ispitni slučajevi za ispitivanje komponenti i cjelokupnog sustava.
			Cilj takvog ispitivanje je ranije otkrivanje i jednostavnije ispravljanje neispravnih i nepoželjnih ponašanja aplikacije te kao provjera zadovoljava li aplikacija obrasce uporabe.
	
			\clearpage
			\subsection{Ispitivanje komponenti}
%			\textit{Potrebno je provesti ispitivanje jedinica (engl. unit testing) nad razredima koji implementiraju temeljne funkcionalnosti. Razraditi \textbf{minimalno 6 ispitnih slučajeva} u kojima će se ispitati redovni slučajevi, rubni uvjeti te izazivanje pogreške (engl. exception throwing). Poželjno je stvoriti i ispitni slučaj koji koristi funkcionalnosti koje nisu implementirane. Potrebno je priložiti izvorni kôd svih ispitnih slučajeva te prikaz rezultata izvođenja ispita u razvojnom okruženju (prolaz/pad ispita). }
			\newcounter{testcase}
			\refstepcounter{testcase}
			
			\indent Ispitivanje komponenti se kontinuirano prati koristeći GitLab CI/CD\footnote{\url{https://docs.gitlab.com/ee/ci/}}. 
			U nastavku su prikazani ispitni slučajevi komponenti koji se prate, njihovi očekivani i trenutni rezultati, a na kraju potpoglavlja je prikazan rezultat izvođenja ispita u razvojnom okruženju JUnit\footnote{\url{https://junit.org/}}.  \\
			
			\noindent \textbf{Ispitni slučaj \thetestcase: Registracija korisnika sprema korisnika u bazu podataka} \\
			\noindent \textbf{Ulaz:}
			\begin{packed_enum}
				\item Pozivanje metode za registraciju korisnika s ispravnim podacima.
			\end{packed_enum}
			\noindent \textbf{Očekivani rezultat:}
			\begin{packed_enum}
				\item[1.a] Korisnik je spremljen u bazu podataka.
			\end{packed_enum}
			\noindent \textbf{Izvorni kod:}
			\begin{listing}[H]
\begin{minted}{java}
@Test
void testRegistrationSavesUser() {
    UserDTO firstUser = new UserDTO("username", "email@email.com", "password");

    userController.signUp(firstUser);

    // User was saved
    verify(userRepository, times(1))
            .save(any());
}
\end{minted}
				\caption{Izvorni kod za ispitni slučaj \thetestcase}
				\label{unittestcase1}
			\end{listing}
			\noindent \textbf{Rezultat:} Očekivani rezultat je zadovoljen. Registracija korisnika s ispravnim podacima je uspješno spremila tog korisnika u bazu podataka. 
			\refstepcounter{testcase}
			\clearpage

			\noindent \textbf{Ispitni slučaj \thetestcase: Registracija dva korisnika s istom e-mail adresom} \\
			\noindent \textbf{Ulaz:}
			\begin{packed_enum}
				\item Pozivanje metode za registraciju korisnika.
				\item Pozivanje metode za registraciju korisnika s istom e-mail adresom.
			\end{packed_enum}
			\noindent \textbf{Očekivani rezultat:}
			\begin{packed_enum}
				\item[1.a] Korisnik je spremljen u bazu podataka.
				\item[2.a] Izazvana je iznimka zbog duplicirane e-mail adrese.
			\end{packed_enum}
			\noindent \textbf{Izvorni kod:}
			\begin{listing}[H]
\begin{minted}{java}
@Test
void testRegistrationWithDuplicateEmailAddress() {
	UserDTO firstUser = new UserDTO("username1", "email@email.com"
                                         "password1");
	UserDTO secondUser = new UserDTO("username2", "email@email.com",
                                         "password2");
		
	userController.signUp(firstUser);
	
	// First user was saved
	verify(userRepository, times(1))
	    .save(any());
	
	// Second user causes an exception
	assertThrows(Exception.class, () -> userController.signUp(secondUser));
}
\end{minted}
				\caption{Izvorni kod za ispitni slučaj \thetestcase}
				\label{testRegistrationWithDuplicateEmailAddress}
			\end{listing}
			\noindent \textbf{Rezultat:} Očekivani rezultat je zadovoljen. Registracija prvog korisnika uspješno sprema tog korisnika u bazu podataka, dok registracija drugog korisnika rezultira iznimkom. 
			\refstepcounter{testcase}
			\clearpage

			\noindent \textbf{Ispitni slučaj \thetestcase: Registracija dva korisnika s istim korisničkim imenom} \\
			\noindent \textbf{Ulaz:}
			\begin{packed_enum}
				\item Pozivanje metode za registraciju korisnika.
				\item Pozivanje metode za registraciju korisnika s istim korisničkim imenom.
			\end{packed_enum}
			\noindent \textbf{Očekivani rezultat:}
			\begin{packed_enum}
				\item[1.a] Korisnik je spremljen u bazu podataka.
				\item[2.a] Izazvana je iznimka zbog dupliciranog korisničkog imena.
			\end{packed_enum}
			\noindent \textbf{Izvorni kod:}
			\begin{listing}[H]
\begin{minted}{java}
@Test
void testRegistrationWithDuplicateUsername() {
	UserDTO firstUser = new UserDTO("username", "email1@email.com"
                                         "password1");
	UserDTO secondUser = new UserDTO("username", "email2@email.com",
                                         "password2");
		
	userController.signUp(firstUser);
	
	// First user was saved
	verify(userRepository, times(1))
	    .save(any());
	
	// Second user causes an exception
	assertThrows(Exception.class, () -> userController.signUp(secondUser));
}
\end{minted}
				\caption{Izvorni kod za ispitni slučaj \thetestcase}
				\label{test5}
			\end{listing}
			\noindent \textbf{Rezultat:} Očekivani rezultat je zadovoljen. Registracija prvog korisnika uspješno sprema tog korisnika u bazu podataka, dok registracija drugog korisnika rezultira iznimkom. 
			\refstepcounter{testcase}
			\clearpage
			
			\noindent \textbf{Ispitni slučaj \thetestcase: Korisnik smije dohvatiti vlastite osobne podatke} \\
			\noindent \textbf{Ulaz:}
			\begin{packed_enum}
				\item Pozivanje metode za dohvat osobnih podataka za trenutno prijavljenog korisnika.
			\end{packed_enum}
			\noindent \textbf{Očekivani rezultat:}
			\begin{packed_enum}
				\item[1.a] Vraćeni su osobni podaci prijavljenog korisnika.
			\end{packed_enum}
			\noindent \textbf{Izvorni kod:}

			\begin{listing}[H]
\begin{minted}{java}
@Test
void testUserCanFetchItsOwnData() {
    userController.getUserInformation("logged_in_user",
            () -> "logged_in_user");

    // Verify interaction with database
    verify(userRepository, times(1))
            .findByUsername(anyString());
}
\end{minted}
				\caption{Izvorni kod za ispitni slučaj \thetestcase}
				\label{test2}
			\end{listing}
			\noindent \textbf{Rezultat:} Očekivani rezultat je zadovoljen. Dohvaćeni su osobni podaci prijavljenog korisnika.
			\refstepcounter{testcase}
			\clearpage

			\noindent \textbf{Ispitni slučaj \thetestcase: Korisnik ne smije moći dohvatiti osobne podatke drugog korisnika} \\
			\noindent \textbf{Ulaz:}
			\begin{packed_enum}
				\item Pozivanje metode za dohvat osobnih podataka za korisnika koji nije trenutno prijavljen.
			\end{packed_enum}
			\noindent \textbf{Očekivani rezultat:}
			\begin{packed_enum}
				\item[1.a] Izazvana je iznimka zbog pokušaja dohvata tuđih podataka.
			\end{packed_enum}
			\noindent \textbf{Izvorni kod:}

			\begin{listing}[H]
\begin{minted}{java}
@Test
void testUserCannotFetchOthersData() {
    assertThrows(Exception.class, () ->
            userController.getUserInformation("different_user", 
                    () -> "logged_in_user"));

    // Verify no interaction with database
    verify(userRepository, times(0))
            .findByUsername(anyString());
}
\end{minted}
				\caption{Izvorni kod za ispitni slučaj \thetestcase}
				\label{test2}
			\end{listing}
			\noindent \textbf{Rezultat:} Očekivani rezultat je zadovoljen. Izazvana je iznimka pri pokušaju dohvata osobnih podataka drugog korisnika.
			\refstepcounter{testcase}
			\clearpage

			\noindent \textbf{Ispitni slučaj \thetestcase: Dohvat podataka o postojećem kontejneru} \\
			\noindent \textbf{Ulaz:}
			\begin{packed_enum}
				\item Pozivanje metode za dohvat podataka o kontejneru s valjanim identifikatorom kontejnera.
			\end{packed_enum}
			\noindent \textbf{Očekivani rezultat:}
			\begin{packed_enum}
				\item[1.a] Vraćeni su podaci o kontejneru s tim identifikatorom.
			\end{packed_enum}
			\noindent \textbf{Izvorni kod:}

			\begin{listing}[H]
\begin{minted}{java}
@Test
void testFetchingInformationAboutExistingWasteContainer() {
    wasteContainerController.findByID(EXISTING_WASTE_CONTAINER_ID);

    // Verify interaction with database
    verify(wasteContainerRepository, times(1))
            .findById(EXISTING_WASTE_CONTAINER_ID);
}
\end{minted}
				\caption{Izvorni kod za ispitni slučaj \thetestcase}
				\label{test2}
			\end{listing}
			\noindent \textbf{Rezultat:} Očekivani rezultat je zadovoljen. Dohvaćeni su podaci o kontejneru s tim identifikatorom.
			\refstepcounter{testcase}
			\clearpage
			
			\noindent \textbf{Ispitni slučaj \thetestcase: Dohvat podataka o nepostojećem kontejneru} \\
			\noindent \textbf{Ulaz:}
			\begin{packed_enum}
				\item Pozivanje metode za dohvat podataka o kontejneru s nevaljanim identifikatorom kontejnera.
			\end{packed_enum}
			\noindent \textbf{Očekivani rezultat:}
			\begin{packed_enum}
				\item[1.a] Izazvana je iznimka zbog pokušaja dohvata podataka o kontejneru koji ne postoji.
			\end{packed_enum}
			\noindent \textbf{Izvorni kod:}

			\begin{listing}[H]
\begin{minted}{java}
@Test
void testFetchingInformationAboutNonExistantWasteContainer() {
    assertThrows(Exception.class,
            () -> 
	       wasteContainerController.findByID(NON_EXISTANT_WASTE_CONTAINER_ID));

    // Verify interaction with database
    verify(wasteContainerRepository, times(1))
            .findById(NON_EXISTANT_WASTE_CONTAINER_ID);
}
\end{minted}
				\caption{Izvorni kod za ispitni slučaj \thetestcase}
				\label{test3}
			\end{listing}
			\noindent \textbf{Rezultat:} Očekivani rezultat je zadovoljen. Dohvat podataka o nepostojećem kontejneru je izazvao iznimku.
			\refstepcounter{testcase}

\begin{listing}[H]
	\begin{minted}[linenos,breaklines,fontsize=\footnotesize,frame=lines]{text}
[INFO] --- maven-surefire-plugin:2.22.2:test (default-test) @ manje-smece-vise-srece-backend ---
[INFO] 
[INFO] -------------------------------------------------------
[INFO]  T E S T S
[INFO] -------------------------------------------------------
[INFO] Running hr.fer.opp.bashcrash.manjesmecevisesrece.rest.WasteContainerControllerTest
[INFO] Tests run: 2, Failures: 0, Errors: 0, Skipped: 0, Time elapsed: 0.656 s - in hr.fer.opp.bashcrash.manjesmecevisesrece.rest.WasteContainerControllerTest
[INFO] Running hr.fer.opp.bashcrash.manjesmecevisesrece.rest.UserControllerTest
[INFO] Tests run: 5, Failures: 0, Errors: 0, Skipped: 0, Time elapsed: 0.118 s - in hr.fer.opp.bashcrash.manjesmecevisesrece.rest.UserControllerTest
[INFO] 
[INFO] Results:
[INFO] 
[INFO] Tests run: 7, Failures: 0, Errors: 0, Skipped: 0
[INFO] 
[INFO] ------------------------------------------------------------------------
[INFO] BUILD SUCCESS
[INFO] ------------------------------------------------------------------------
[INFO] Total time:  6.019 s
[INFO] Finished at: 2020-01-02T23:31:26+01:00
[INFO] ------------------------------------------------------------------------
	\end{minted}
	\caption{Prikaz rezultata izvođenja ispita u razvojnom okruženju}
\end{listing}
			\clearpage

			\subsection{Ispitivanje sustava}
			
		 \indent Ispitivanje sustava prati se ručno upotrebom Selenium IDE alata integriranog u Google Chrome preglednik\footnote{\url{https://selenium.dev/selenium-ide/}}. 
		 Ispitni slučajevi ispituju obrasce uporabe UC4, UC5, UC6, UC7, UC8, UC11, UC12 i UC13.
	  	 U nastavku su prikazani koraci ispitivanja, njihovi očekivani i trenutni rezultati te programski isječci ispitnih slučajeva.
		 \newline
			
%			 \textit{Potrebno je provesti i opisati ispitivanje sustava koristeći radni okvir Selenium\footnote{\url{https://www.seleniumhq.org/}}. Razraditi \textbf{minimalno 4 ispitna slučaja} u kojima će se ispitati redovni slučajevi, rubni uvjeti te poziv funkcionalnosti koja nije implementirana/izaziva pogrešku kako bi se vidjelo na koji način sustav reagira kada nešto nije u potpunosti ostvareno. Ispitni slučaj se treba sastojati od ulaza (npr. korisničko ime i lozinka), očekivanog izlaza ili rezultata, koraka ispitivanja i dobivenog izlaza ili rezultata.\\ }
%			 
%			 \textit{Izradu ispitnih slučajeva pomoću radnog okvira Selenium moguće je provesti pomoću jednog od sljedeća dva alata:}
%			 \begin{itemize}
%			 	\item \textit{dodatak za preglednik \textbf{Selenium IDE} - snimanje korisnikovih akcija radi automatskog ponavljanja ispita	}
%			 	\item \textit{\textbf{Selenium WebDriver} - podrška za pisanje ispita u jezicima Java, C\#, PHP koristeći posebno programsko sučelje.}
%			 \end{itemize}
%		 	 \textit{Detalji o korištenju alata Selenium bit će prikazani na posebnom predavanju tijekom semestra.}
			\noindent \textbf{Ispitni slučaj \thetestcase: Prijava i odjava korisnika} \\
			\noindent \textbf{Ulaz:}
			\begin{packed_enum}
				\item Otvaranje početne stranice u web pregledniku.
				\item Pritisak na ikonu korisnika u gornjem desnom kutku.
				\item Odabir opcije "Prijava" iz padajućeg izbornika.
				\item Upisivanje korisničkog imena i lozinke te pritisak gumba "Prijava".
				\item Pritisak na ikonu korisnika u gornjem desnom kutku.
				\item Odabir opcije "Odjava" iz padajućeg izbornika.
			\end{packed_enum}
			\noindent \textbf{Očekivani rezultat:}
			\begin{packed_enum}
				\item[1.a] Prikazuje se početna stranica s kartom.
				\item[1.b] Na karti su prikazani kontejneri koji su uneseni u aplikaciju.
				\item[2.\ \ ] Prikazuje se padajući izbornik s opcijama "Prijava" i "Registracija".
				\item[3.\ \ ] Prikazuje se forma za prijavu korisnika s tekstualnim poljima za unos korisničkog imena i lozinke.
				\item[4.a] Prikazuje se početna stranica s dodatnim gumbima sa strane.
				\item[4.b] Korisnik je uspješno prijavljen.
				\item[5.\ \ ] Prikazuje se padajući izbornik s opcijama "Postavke" i "Odjava".
				\item[6.a] Prikazuje se početna stranica bez dodatnih gumbi sa strane.
				\item[6.b] Korisnik je odjavljen.
			\end{packed_enum}
			\noindent \textbf{Izvorni kod:}

\begin{minted}{java}
    //otvaranje pocetne stranice
driver.get("http://localhost:3000/");
driver.manage().window().setSize(new Dimension(1051, 806));

    //prijava
driver.findElement(By.cssSelector(".dropdown-toggle")).click();
driver.findElement(By.linkText("Prijava")).click();
    
    //upis korisnickog imena
driver.findElement(By.id("username")).click();
driver.findElement(By.id("username")).sendKeys("korisnik");
    
    //upis lozinke
driver.findElement(By.id("password")).click();
driver.findElement(By.id("password")).sendKeys("korisnik_lozinka");
    
    //potvrda
driver.findElement(By.cssSelector("div:nth-child(5) > .btn")).click();

    //odjava
driver.findElement(By.cssSelector(".dropdown-toggle")).click();
driver.findElement(By.linkText("Odjava")).click();

    //zatvaranje
driver.close();
 
\end{minted}
			\begin{listing}[H]
				\caption{Izvorni kod za ispitni slučaj \thetestcase}
				\label{test3}
			\end{listing}
			\noindent \textbf{Rezultat:} Očekivani rezultat je ostvaren. Korisnik je uspješno prijavljen i odjavljen.
				\begin{figure}[H]
            					\includegraphics[width=\linewidth]{slike/selenium/LoginLogout.png}
            					\centering
            					\caption{Rezultat za ispitni slučaj 8}
            					\label{fig:test 8}
            		            \end{figure}
			\refstepcounter{testcase}
			\clearpage
		

			\noindent \textbf{Ispitni slučaj \thetestcase: Pregled osobnih podataka} \\
			\noindent \textbf{Ulaz:}
			\begin{packed_enum}
				\item Otvaranje početne stranice u web pregledniku te prijava u sustav\footnotemark.
				\item Pritisak na ikonu korisnika u gornjem desnom kutku.
				\item Odabir opcije "Postavke" iz padajućeg izbornika.
				\item Odjava iz sustava\footnotemark[\value{footnote}].

				\footnotetext{Postupak prijave i odjave korisnika u sustav je prikazan u ispitnom slučaju 8.}
			\end{packed_enum}
			\noindent \textbf{Očekivani rezultat:}
			\begin{packed_enum}
				\item[1.a] Prikazuje se početna stranica.
				\item[1.b] Korisnik je prijavljen.
				\item[2.\ \ ] Prikazuje se padajući izbornik s opcijama "Postavke" i "Odjava".
				\item[3.\ \ ] Prikazuju se osobni podaci korisnika.
				\item[4.\ \ ] Korisnik je odjavljen iz sustava.
			\end{packed_enum}
			\noindent \textbf{Izvorni kod:}

			\begin{listing}[H]

\begin{minted}{java}
    //prijava
    
    //klik na padajuci izbornik
driver.findElement(By.cssSelector(".dropdown-toggle")).click();
    
    //klik na korisnika
driver.findElement(By.linkText("korisnik")).click();

    //pregled osobnih podataka

    //odjava
\end{minted}
				\caption{Izvorni kod za ispitni slučaj \thetestcase}
				\label{test3}
			\end{listing}
			\noindent \textbf{Rezultat:} Očekivani rezultat je ostvaren. Pregled osobnih podataka bio je uspješan. 
			
				\begin{figure}[H]
            					\includegraphics[width=\linewidth]{slike/selenium/PersonalDataTest.png}
            					\centering
            					\caption{Rezultat za ispitni slučaj 9}
            					\label{fig:test 9}
            		            \end{figure}
			\refstepcounter{testcase}
			\clearpage

			\noindent \textbf{Ispitni slučaj \thetestcase: Komentiranje kontejnera} \\
			\noindent \textbf{Ulaz:}
			\begin{packed_enum}
				\item Otvaranje početne stranice u web pregledniku te prijava u sustav\footnotemark.
				\item Pritisak na ikonu povećala u lijevom izborniku.
				\item Upisivanje ispravnog identifikatora kontejnera, pritisak na ikonu povećala.
				\item Pritisak na ikonu kontejnera s tim identifikatorom.
				\item Upisivanje komentara kontejnera, ocjenjivanje kontejnera i pritisak na ikonu diskete.
				\item Odjava iz sustava\footnotemark[\value{footnote}].

				\footnotetext{Postupak prijave i odjave korisnika u sustav je prikazan u ispitnom slučaju 8.}
			\end{packed_enum}
			\noindent \textbf{Očekivani rezultat:}
			\begin{packed_enum}
				\item[1.a] Prikazuje se početna stranica.
				\item[1.b] Korisnik je prijavljen.
				\item[2.\ \ ] Prikazuje se tekstualno polje za upis identifikatora kontejnera.
				\item[3.\ \ ] Odabrani kontejner se centrira na karti.
				\item[4.\ \ ] Prikazuje se stranica kontejnera s objavljenim recenzijama tog kontejnera.
				\item[5.a] Komentar je objavljen.
				\item[5.b] Ocjena je zabilježena.
				\item[6.\ \ ] Korisnik je odjavljen iz sustava.
			\end{packed_enum}
			\noindent \textbf{Izvorni kod:}
\begin{minted}{java}

    //prijava

    //klik na gumb za trazenje kontejnera
driver.findElement(By.cssSelector(".btn:nth-child(1) path")).click();
    {
      WebElement element = driver.
        findElement(By.cssSelector(".btn:nth-child(1)"));
      Actions builder = new Actions(driver);
      builder.moveToElement(element).perform();
    }
    
    //upisivanje id-a kontejnera
driver.findElement(By.cssSelector(".form-control")).click();
driver.findElement(By.cssSelector(".form-control")).sendKeys("93");
    
    //potvrda
driver.findElement(By.id("addon-wrapping")).click();

    //klik na kontejner
driver.
    findElement(By.cssSelector("div:nth-child(2) > div > svg > path")).
        click();
    
    //upis ocjene
driver.findElement(By.name("userOcjena")).click();
driver.findElement(By.name("userOcjena")).sendKeys("3");

    //upis komentara
driver.findElement(By.name("userKomentar")).click();
driver.findElement(By.name("userKomentar")).
    sendKeys("Kontejner je u prosjecnom stanju.");
    
    //potvrda
driver.findElement(By.cssSelector("div:nth-child(6) svg")).click();

    //odjava
\end{minted}
			\begin{listing}[H]

				\caption{Izvorni kod za ispitni slučaj \thetestcase}
				\label{test3}
			\end{listing}
			\noindent \textbf{Rezultat:} Očekivani rezultat je ostvaren. Ocjena i komentar su pohranjeni.
			
			\begin{figure}[H]
            					\includegraphics[width=\linewidth]{slike/selenium/containercomment.png}
            					\centering
            					\caption{Rezultat za ispitni slučaj 10, prvi dio}
            					\label{fig:test 10 p1}
            		            \end{figure}
			
			
			\begin{figure}[H]
            					\includegraphics[width=\linewidth]{slike/selenium/containercomment2.png}
            					\centering
            					\caption{Rezultat za ispitni slučaj 10, drugi dio}
            					\label{fig:test 10 p2}
            		            \end{figure}
			\refstepcounter{testcase}
			
			

			\clearpage

			\noindent \textbf{Ispitni slučaj \thetestcase: Upravljanje kontejnerima} \\
			\noindent \textbf{Ulaz:}
			\begin{packed_enum}
				\item Otvaranje početne stranice u web pregledniku te prijava u sustav\footnotemark.
				\item Pritisak na ikonu kontejnera u lijevom izborniku.
				\item Pritisak na ikonu plusa.
				\item Upisivanje podataka o kontejneru i pritisak na ikonu diskete.
				\item Pritisak na ikonu kante za smeće u kartici novostvorenog kontejnera.
				\item Odjava iz sustava\footnotemark[\value{footnote}].

				\footnotetext{Postupak prijave i odjave korisnika u sustav je prikazan u ispitnom slučaju 8.}
			\end{packed_enum}
			\noindent \textbf{Očekivani rezultat:}
			\begin{packed_enum}
				\item[1.a] Prikazuje se početna stranica.
				\item[1.b] Korisnik je prijavljen.
				\item[2.\ \ ] Prikazuje se popis kontejnera koji su uneseni u aplikaciju.
				\item[3.\ \ ] Prikazuje se forma za unos podataka o novom kontejneru.
				\item[4.a] Kontejner je spremljen u bazu podataka.
				\item[4.b] Prikazuje se popis kontejnera koji sadrži novostvoreni kontejner.
				\item[5.a] Kontejner je izbrisan iz baze podataka.
				\item[5.b] Prikazuje se popis kontejnera koji ne sadrži obrisani kontejner.
				\item[6.\ \ ] Korisnik je odjavljen iz sustava.
			\end{packed_enum}
			\noindent \textbf{Izvorni kod:}

		
\begin{minted}{java}
    //prijava
 
    //klik na gumb za trazenje kontejnera
driver.findElement(By.cssSelector(".btn:nth-child(1) > svg")).click();

    //klik na gumb kontejnera    
driver.
    findElement(By.cssSelector(".btn:nth-child(5) > svg")).click();
    
    //klik na gumb za dodavanje kontejnera
driver.
    findElement(By.cssSelector("div:nth-child(1) > svg > path")).click();
    
    //unos podataka za kontejner i potvrda unosa
driver.findElement(By.name("adresa")).click();
driver.findElement(By.name("adresa")).sendKeys("Savska Ulica 1");
driver.
    findElement(By.cssSelector("div:nth-child(2) > button path")).
        click();
        
    //brisanje kontejnera 
driver.
findElement(By.cssSelector("div:nth-child(5) svg:nth-child(2) > path")).
    click();
    
    //odjava
\end{minted}

	\begin{listing}[H]
				\caption{Izvorni kod za ispitni slučaj \thetestcase}
				\label{test3}
			\end{listing}
			\noindent \textbf{Rezultat:} Očekivani rezultat je ostvaren. Upravljanje kontejnerima bilo je uspješno.  
			
			\begin{figure}[H]
            					\includegraphics[width=\linewidth]{slike/selenium/containermanagement.png}
            					\centering
            					\caption{Rezultat za ispitni slučaj 11, prvi dio}
            					\label{fig:test 11 p1}
            \end{figure}
            
            \begin{figure}[H]
            					\includegraphics[width=\linewidth]{slike/selenium/containermanagement2.png}
            					\centering
            					\caption{Rezultat za ispitni slučaj 11, drugi dio}
            					\label{fig:test 11 p2}
            \end{figure}
			
			\refstepcounter{testcase}
			\eject 
		
		
		\section{Dijagram razmještaja}
			
			% \textbf{\textit{dio 2. revizije}}
			
			 % \textit{Potrebno je umetnuti \textbf{specifikacijski} dijagram razmještaja i opisati ga. Moguće je umjesto specifikacijskog dijagrama razmještaja umetnuti dijagram razmještaja instanci, pod uvjetom da taj dijagram bolje opisuje neki važniji dio sustava.}
			
			Dijagrami razmještaja opisuju topologiju sustava te su usredotočeni na odnos sklopovskih i programskih dijelova. Sustav je baziran na arhitekturi "klijent - poslužitelj". Klijenti preko web preglednika pristupaju web aplikaciji dok se na poslužiteljskom računalu nalaze web poslužitelj i poslužitelj baze podataka. Komunikacija između klijenta i poslužitelja odvija se preko HTTP veze.

			\begin{figure} [H]
			    \centering
			    \includegraphics[width=1.0\linewidth]{slike/Deployment_Diagram.png}
                \caption{Dijagram razmještaja}
                \label{fig:Dijagram razmještaja}
		    \end{figure}
			\eject
		

		\section{Upute za puštanje u pogon}
		
			% \textbf{\textit{dio 2. revizije}}\\
		
			 % \textit{U ovom poglavlju potrebno je dati upute za puštanje u pogon (engl. deployment) ostvarene aplikacije. Na primjer, za web aplikacije, opisati postupak kojim se od izvornog kôda dolazi do potpuno postavljene baze podataka i poslužitelja koji odgovara na upite korisnika. Za mobilnu aplikaciju, postupak kojim se aplikacija izgradi, te postavi na neku od trgovina. Za stolnu (engl. desktop) aplikaciju, postupak kojim se aplikacija instalira na računalo. Ukoliko mobilne i stolne aplikacije komuniciraju s poslužiteljem i/ili bazom podataka, opisati i postupak njihovog postavljanja. Pri izradi uputa preporučuje se \textbf{naglasiti korake instalacije uporabom natuknica} te koristiti što je više moguće \textbf{slike ekrana} (engl. screenshots) kako bi upute bile jasne i jednostavne za slijediti.}
			
			
			 % \textit{Dovršenu aplikaciju potrebno je pokrenuti na javno dostupnom poslužitelju. Studentima se preporuča korištenje neke od sljedećih besplatnih usluga: \href{https://aws.amazon.com/}{Amazon AWS}, \href{https://azure.microsoft.com/en-us/}{Microsoft Azure} ili \href{https://www.heroku.com/}{Heroku}. Mobilne aplikacije trebaju biti objavljene na F-Droid, Google Play ili Amazon App trgovini.}
			 
		
		   	\noindent \underbar{\textbf{Backend:}}
		   	\begin{packed_enum}
				
					\item \textbf{Instalacija programske podrške}(Java, Apache Maven, PostgreSQL)
    					\begin{packed_item}
                			\item \textbf{Java} (\url{https://www.oracle.com/technetwork/java/javase/downloads/index.html})
                			\item \textbf{Apache Maven} (\url{https://maven.apache.org/install.html})
                			\item \textbf{PostgreSQL} (\url{https://www.postgresql.org/download/})
    		      \end{packed_item}
					\item \textbf{Postavljanje baze podataka} (Stvaranje korisnika te baze podataka)
					    \begin{verbatim}
					    create user testuser;
					    alter user testuser with encrypted password 'password';
					    create database testdb;
					    grant all privileges on database testdb to testuser;
			            \end{verbatim}
			            
    					\begin{figure}[H]
    					\includegraphics[width=\linewidth]{slike/backend/psqlsetup.png}
    					\centering
    					\caption{Postavljanje baze podataka}
    					\label{fig:PSQL-setup}
    		            \end{figure}
    		            
		            \item \textbf{Konfiguracija \textit{pom.xml} i \textit{application.properties} datoteka}
		                \begin{packed_item}
                			\item {U pom.xml potrebno je dodati:} 
                			\begin{verbatim}
					    <dependency>
					        <groupId>org.postresql</groupId>
					        <artifactId>postresql</artifactId>
					   </dependency>
			            \end{verbatim}
                			    
                			\item {U application.properties potrebno je dodati:}
\begin{minted}{properties}
spring.datasource.url = jdbc:postresql://localhost:5432/testdb
spring.datasource.username = testuser
spring.datasource.password = password
\end{minted}




                			
    		            \end{packed_item}
    		            
    		      \item \textbf{Pozicioniranje u radni direktorij backend-a putem cmd terminala}
    		            \begin{figure}[H]
    					\includegraphics[width=\linewidth]{slike/backend/backendposition.png}
    					\centering
    					\caption{Pozicioniranje u radni direktorij}
    					\label{fig:pozicioniranje}
    		            \end{figure}
    		        
    		      \item \textbf{mvn install}
    		            \begin{figure}[H]
    					\includegraphics[width=\linewidth]{slike/backend/backendinstall.png}
    					\centering
    					\caption{mvn install}
    					\label{fig:mvn-install}
    		            \end{figure}
    		            
    		      \item \textbf{mvn package}
    		            \begin{figure}[H]
    					\includegraphics[width=\linewidth]{slike/backend/backendpackage.png}
    					\centering
    					\caption{mvn package}
    					\label{fig:mvn package}
    		            \end{figure}
    		            
    		      \item \textbf{java -jar target/manje-smece-vise-srece-backend-0.0.1-SNAPSHOT.jar}
    		            \begin{figure}[H]
    					\includegraphics[width=\linewidth]{slike/backend/backendjar.png}
    					\centering
    					\caption{Pokretanje}
    					\label{fig:pokretanje}
    		            \end{figure}
  	   
				
				\end{packed_enum}
		   	
		   	\noindent \underbar{\textbf{Frontend:}}
		   		\begin{packed_enum}
				
					\item \textbf{Instalacija programske podrške}(Node.js)
    					\begin{packed_item}
                			\item \textbf{Node.js} (\url{https://nodejs.org/en/download/})
                		
    		      \end{packed_item}
					\item \textbf{Konfiguracija package.json datoteke}
					\begin{packed_item}
                			\item \textbf{U package.json datoteku potrebno je dodati: }
                			\begin{verbatim}"proxy": "http://localhost:8080"
                			\end{verbatim}
                		
    		      \end{packed_item}
					
    					
    		            
    		  \item \textbf{Pozicioniranje u radni direktorij frontenda-a putem cmd terminala}
    		            \begin{figure}[H]
    					\includegraphics[width=\linewidth]{slike/frontend/prva.jpg}
    					\centering
    					\caption{Pozicioniranje u radni direktorij}
    					\label{fig:pozicioniranje2}
    		            \end{figure}
    		            
    		 \item \textbf{npm install}
    		            \begin{figure}[H]
    					\includegraphics[width=\linewidth]{slike/frontend/treca.png}
    					\centering
    					\caption{npm install}
    					\label{fig:npm install}
    		            \end{figure}
    		            
            \item \textbf{npm start}
    		            \begin{figure}[H]
    					\includegraphics[width=\linewidth]{slike/frontend/treca.jpg}
    					\centering
    					\caption{npm start}
    					\label{fig:npm start}
    		            \end{figure}
    		            
    		            \begin{figure}[H]
    					\includegraphics[width=\linewidth]{slike/frontend/cetvrta.png}
    					\centering
    					\caption{Konačno}
    					\label{fig:konacno}
    		            \end{figure}
    		            
    		            
    		            
    		            
		           
  	   
				
				\end{packed_enum}
           
            
            
			
			
			
			\eject 
