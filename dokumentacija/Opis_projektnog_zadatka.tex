\chapter{Opis projektnog zadatka}

		Cilj ovog projekta je razviti programsku podršku za web aplikaciju koja će građanima omogućiti slanje stanja kontejnera djelatnicima komunalnih službi. Aplikacija bi, temeljem zaprimljenih stanja kontejnera, pomogla komunalnim službama da problem odvoza smeća rješavaju na efikasniji i brži način.\\
		
		Prilikom pokretanja sustava, prikazuje se karta s ucrtanim kontejnerima koji se nalaze u blizini te traka za pretraživanje kontejnera.
        Neregistriranom korisniku, odabirom kontejnera na karti, otvaraju se opće informacije o istom; ID kontejnera, adresa te povijest stanja
        koja je vidljiva kroz recenzije i slike koje su korisnici objavljivali.\\

        Osim pregleda kontejnera odabirom na karti, neregistriranim korisnicima ponuđena je opcija pretraživanja
        putem jedinstvenog ID-a dodijeljenog svakom kontejneru. Upisom valjanog ID-a, otvaraju se opće informacije o kontejneru, a u suprotnom korisnik dobiva odgovarajuću
        poruku.\\
        
        Neregistriranom korisniku omogućeno je registriranje u sustav/kreiranje novog računa, a
        za to su potrebni sljedeći podatci:
        \begin{packed_item}
			
			\item  korisničko ime
			\item  lozinka
			\item  email adresa
			
		\end{packed_item}
		
		Registracijom u sustav, korisnik dobiva mogućnost ocjenjivanja i komentiranja stanja kontejnera kao građanin. Korisnik, osim kao građanin, može biti djelatnik ili direktor komunalne službe.\\
		
		\underbar{\textit{Građanin}} je registrirani korisnik koji ima mogućnost ocjenjivanja i komentiranja stanja kontejnera. Osim navedenog, građanin može slikati kontejner i objaviti njegovo stvarno stanje.\\
		
		\underbar{\textit{Direktor}} ima širi spektar mogućnosti rada u aplikaciji. Njemu je omogućeno pregled svih kontejnera, zaposlenika, zona i ruta. Također mu je omogućeno dodavanje novih kontejnera i zaposlenika, te svrstavanje zaposlenika i kontejnera u zone.\\
		
		
		\eject
	