\chapter{Opis projektnog zadatka}

		Cilj ovog projekta je razviti programsku podršku za web aplikaciju koja će građanima omogućiti slanje stanja kontejnera djelatnicima komunalnih službi. Aplikacija bi, temeljem zaprimljenih stanja kontejnera, pomogla komunalnim službama da problem odvoza smeća rješavaju na efikasniji i brži način.\\
		
		Prilikom pokretanja sustava, prikazuje se karta s ucrtanim kontejnerima koji se nalaze u blizini te traka za pretraživanje kontejnera.
        Neregistriranom korisniku, odabirom kontejnera na karti, otvaraju se opće informacije o istom; ID kontejnera, adresa te povijest stanja
        koja je vidljiva kroz recenzije i slike koje su korisnici objavljivali.\\

        Osim pregleda kontejnera odabirom na karti, neregistriranim korisnicima ponuđena je opcija pretraživanja
        putem jedinstvenog ID-a dodijeljenog svakom kontejneru. Upisom valjanog ID-a, otvaraju se opće informacije o kontejneru, a u suprotnom korisnik dobiva odgovarajuću
        poruku.\\
        
        Neregistriranom korisniku omogućeno je registriranje u sustav/kreiranje novog računa, a
        za to su potrebni sljedeći podatci:
        \begin{packed_item}
			
			\item  korisničko ime
			\item  lozinka
			\item  email adresa
			
		\end{packed_item}
		
		Registracijom u sustav, korisnik dobiva mogućnost ocjenjivanja i komentiranja stanja kontejnera kao građanin. Korisnik, osim kao građanin, može biti djelatnik ili direktor komunalne službe.\\
		
		\underbar{\textit{Građanin}} je registrirani korisnik koji ima mogućnost ocjenjivanja i komentiranja stanja kontejnera. Osim navedenog, građanin može slikati kontejner i objaviti njegovo stvarno stanje.\\
		
		
		\eject
		
		
		\section{Primjeri u LaTeXu}
		
		\textit{Ovo potpoglavlje izbrisati.}\\

		U nastavku se nalaze različiti primjeri kako koristiti osnovne funkcionalnosti LaTeXa koje su potrebne za izradu dokumentacije. Za dodatnu pomoć obratiti se asistentu na projektu ili potražiti upute na sljedećim web sjedištima:
		\begin{itemize}
			\item Upute za izradu diplomskog rada u LaTeXu - \url{https://www.fer.unizg.hr/_download/repository/LaTeX-upute.pdf}
			\item LaTeX projekt - \url{https://www.latex-project.org/help/}
			\item StackExchange za Tex - \url{https://tex.stackexchange.com/}\\
		
		\end{itemize} 	


		
		%Ovo poglavlje je potrebno prilikom predaje obrisati
		
		\underbar{podcrtani tekst}, 
		\textbf{podebljani tekst}, 
		\textit{nagnuti tekst}\\
		\normalsize primjer
		\large primjer
		\Large primjer
		\LARGE {primjer}
		\huge {primjer}
		\Huge primjer
		\normalsize
				
		\begin{packed_item}
			
			\item  primjer
			\item  primjer
			\item  primjer
			\item[] \begin{packed_enum}
				
				\item primjer
				\item primjer
			\end{packed_enum}
			
		\end{packed_item}
		
		\noindent primjer url-a: \url{https://www.fer.unizg.hr/predmet/opp/projekt}
		
		
		\begin{longtabu} to \textwidth {|X[8, l]|X[8, l]|X[16, l]|} %definicija sirine polja
			
			\hline \multicolumn{3}{|c|}{\textbf{naslov unutar tablice}}	 \\[3pt] \hline
			\endfirsthead
			
			\hline \multicolumn{3}{|c|}{\textbf{naslov unutar tablice}}	 \\[3pt] \hline
			\endhead
			
			\hline 
			\endlastfoot
			
			\rowcolor{LightGreen}IDKorisnik & INT	&  	Lorem ipsum dolor sit amet, consectetur adipiscing elit, sed do eiusmod  	\\ \hline
			korisnickoIme	& VARCHAR &   	\\ \hline 
			email & VARCHAR &   \\ \hline 
			ime & VARCHAR	&  		\\ \hline 
			\cellcolor{LightBlue} primjer	& VARCHAR &   	\\ \hline 
			
			
		\end{longtabu}
		

		\begin{table}[H]
			
			
			
			\begin{longtabu} to \textwidth {|X[8, l]|X[8, l]|X[16, l]|} %definicija sirine polja
				
				\hline 
				\endfirsthead
				
				\hline 
				\endhead
				
				\hline 
				\endlastfoot
				
				\rowcolor{LightGreen}IDKorisnik & INT	&  	Lorem ipsum dolor sit amet, consectetur adipiscing elit, sed do eiusmod  	\\ \hline
				korisnickoIme	& VARCHAR &   	\\ \hline 
				email & VARCHAR &   \\ \hline 
				ime & VARCHAR	&  		\\ \hline 
				\cellcolor{LightBlue} primjer	& VARCHAR &   	\\ \hline 
				
				
			\end{longtabu}
	
			\caption{\label{tab:referencatablica} Naslov ispod tablice.}
		\end{table}
		
		\begin{figure}[H]
			\includegraphics[scale=0.4]{slike/aktivnost.PNG}
			\centering
			\caption{Primjer slike s potpisom}
			\label{fig:promjene}
		\end{figure}
		
		\begin{figure}[H]
			\includegraphics[width=\linewidth]{slike/aktivnost.PNG}
			\caption{Primjer slike s potpisom 2}
			\label{fig:promjene2}
		\end{figure}
		
		
		
		\eject
		
	